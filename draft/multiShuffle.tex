\documentclass[12pt]{amsart}

%\usepackage[notcite,notref]{showkeys}
\usepackage{tikz}
%\usepackage{bbm,bm}
%\usepackage{rotating}
\usepackage{amssymb}
\usepackage{amsmath}
\usepackage{comment}
\usepackage{bm}
\usepackage[all]{xy}
%\usepackage{todonotes}

\sloppy
\flushbottom

\usepackage{color}
\newcommand{\blue}[1]{\textcolor{blue}{#1}}
\newcommand{\red}[1]{\textcolor{red}{#1}}
\newcommand{\green}[1]{\textcolor{green!75!black}{#1}}

\def\Pda{\Psi_{21}(\alpha)}

\usepackage[bookmarks=false,colorlinks,linkcolor=blue,citecolor=green]{hyperref}


%%%%%%%%%%%%%%
\usepackage{xargs}
\usepackage[colorinlistoftodos,prependcaption,textsize=tiny,linecolor=red,backgroundcolor=red!25,bordercolor=red]{todonotes}
\setlength{\marginparwidth}{2cm}
\newcommandx{\nantel}[2][1=]{\todo[linecolor=brown,backgroundcolor=red!25,bordercolor=blue,#1]{#2 ---Nantel}}
\newcommandx{\jc}[2][1=]{\todo[linecolor=purple,backgroundcolor=blue!25,bordercolor=red,#1]{#2 ---JC}}
\newcommandx{\john}[2][1=]{\todo[linecolor=purple,backgroundcolor=green!25,bordercolor=red,#1]{#2 ---John}}


%%---- page geometry & headings
%%%%%%%%%%%%%%%%%%%%%%%%%%%%%%%% 
\addtolength{\textheight}{48pt}
\addtolength{\voffset}{-24pt}
\addtolength{\textwidth}{86pt}
\addtolength{\hoffset}{-42pt}
%%%%%%%%%%%%%%%%%%%%%%%%%%%%%%%%

\pagestyle{headings} % empty}

\date{\today}

%-------------------------------------------------------------------
% ENVIRONMENTS
%-------------------------------------------------------------------
\theoremstyle{plain}
\newtheorem{theorem}{Theorem}[section]
\newtheorem{corollary}[theorem]{Corollary}
\newtheorem{lemma}[theorem]{Lemma}
\newtheorem{proposition}[theorem]{Proposition}
\newtheorem{conjecture}{\red{Conjecture}}
\newtheorem{property}{\red{Property}}

\theoremstyle{definition}
\newtheorem{example}[theorem]{Example} 
\newtheorem{remark}[theorem]{Remark} 
\newtheorem{definition}[theorem]{Definition} 

\theoremstyle{remark}
\newtheorem*{associativity}{Associativity}
%\newtheorem*{unitality}{Unitality}
%\newtheorem*{naturality}{Naturality}
\newtheorem*{compatibility}{Compatibility}
%\newtheorem*{connectedness}{Connectedness}
%\newtheorem*{multiplicativity}{Multiplicativity}
%\newtheorem*{warning}{Warning}

 
 
% TO SEE SPECIAL COMMENT "uncomment" the line bellow
% make sure that "\excludecomment{moredetails}" is the "commented" 
% TO HIDE SPECIAL COMMENT do the opposite as above
\specialcomment{moredetails}{\begingroup\color{red}}{\endgroup}
%\excludecomment{moredetails}


% SET: I,J,K,L
% Set composition:  A,B,C,D
% Parts O,P,Q,R,T,E,F

%-------------------------------------------------------------------
% MACROS
%-------------------------------------------------------------------
\newcommand{\draftnote}[1]{\marginpar{\raggedright\textsf{\hspace{0pt}\red{#1}}}}

\newcommand{\Vinc}{\begin{turn}{90}$\subseteq$\end{turn}} 
\newcommand{\Veq}{\begin{turn}{90}$=$\end{turn}}

\def\id{\mathrm{id}}
\def\Id{{\bf 1}}
\DeclareMathOperator{\coker}{coker}
\DeclareMathOperator{\supp}{supp}
\DeclareMathOperator{\End}{End}

\def\llb{{[\![}}
\def\rrb{{]\!]}}


\newcommand{\abs}[1]{\lvert#1\rvert} %for absolute value
\newcommand{\pmat}[1]{\begin{pmatrix}#1\end{pmatrix}} %for round matrix

\newcommand{\larc}[1]{\hspace{-.4ex}\overset{#1}{\frown}\hspace{-.4ex}}
\newcommand{\slarc}[1]{\overset{#1}{\frown}}

\newcommand{\gel}{>_{\ell}}
\newcommand{\lel}{<_{\ell}}

\newcommand{\one}{\mathbf{1}}
\newcommand{\euler}{\mathbf{e}}

\let\onto=\twoheadrightarrow
\let\into=\hookrightarrow
\let\map=\xrightarrow
\newcommand{\xyinc}{\ar@{^{(}->}}
\newcommand{\mapsby}[1]{ \overset{#1}{\longmapsto}}

\newcommand{\type}[2]{\mathsf T_{#1}(#2)}

\def\field{\Bbbk}
\newcommand{\Sr}{\mathrm{S}} %symmetric group

\newcommand{\rH}{\mathrm{H}}
\newcommand{\rL}{{\mathbf l}}
\newcommand{\rG}{{\mathbf g}}
\newcommand{\rD}{\mathrm{D}}

\newcommand{\bL}{\mathbf L}
\newcommand{\bE}{\mathbf E}
\newcommand{\bH}{\mathbf H}
\newcommand{\bh}{\mathbf h}
\newcommand{\bk}{\mathbf k}
\newcommand{\bK}{\mathbf K}
\newcommand{\bPi}{\mathbf \Pi}
\newcommand{\bG}{\mathbf{G}} 
\newcommand{\bp}{\mathbf p}
\newcommand{\bq}{\mathbf q}
\newcommand{\bmm}{\mathbf m}%\bm exists
\newcommand{\bd}{\mathbf d}

\newcommand{\Fc}{\mathcal{F}}
\newcommand{\Tc}{\mathcal{T}}
\newcommand{\Kc}{\mathcal{K}}
\newcommand{\Kcb}{\overline{\Kc}}

\newcommand{\Nb}{\mathbb{N}}
\newcommand{\Qb}{\mathbb{Q}}
\newcommand{\Cb}{\mathbb{C}}
\newcommand{\Zb}{\mathbb{Z}}
\newcommand{\Fb}{\mathbb{F}}
\newcommand{\Fq}{\Fb_q}


\newcommand{\Qvar}{{\bm q}}
\newcommand{\Zvar}{{\bm z}}


\newcommand{\Nf}{\mathfrak{n}}
\newcommand{\bNf}{\overline{\Nf}}
\newcommand{\Mf}{\mathfrak{m}}
\newcommand{\bMf}{\overline{\Mf}}

\newcommand{\GL}{\mathrm{GL}}
\newcommand{\Un}{\mathrm{U}}
\newcommand{\Mat}{\mathrm{M}}
\newcommand{\Tri}{\mathrm{T}}

\newcommand{\FC}{\bm{\mathrm{f}}}
\newcommand{\CF}{\bm{\mathrm{cf}}}
\newcommand{\SC}{\bm{\mathrm{scf}}}

\newcommand{\bdot}{\bm\cdot}

\newcommand{\qand}{\quad\text{and}\quad}
\newcommand{\qor}{\quad\text{or}\quad}
\newcommand{\qqand}{\qquad\text{and}\qquad}
\newcommand{\qqor}{\qquad\text{or}\qquad}

%\newcommand{\jc}[1]{\todo[color=orange!30]{#1 \\ \hfill --- J-C.}}
%\newcommand{}[1]{\todo[color=red!30]{#1 \\ \hfill --- N.}}
\setlength{\marginparwidth}{2cm}

\newcommand{\PF}{\mathcal{PF}}
\newcommand\area{\mathop{\mathrm{area}}}
\newcommand{\dinv}{\mathop{\mathrm{dinv}}}
\newcommand{\ides}{\mathop{\mathrm{ides}}}
\newcommand{\des}{\mathop{\mathrm{des}}}
\newcommand{\desc}{\mathop{\mathrm{desc}}}
\newcommand{\ank}{\mathop{\mathrm{ank}}}
\newcommand{\ret}{\mathop{\mathrm{ret}}}
\newcommand{\comp}{\mathop{\mathrm{comp}}}
\newcommand{\ps}{\mathop{\mathrm ps^1}}

% math commands
\newcommand{\set}[2]{\left\{ #1 \;\middle|\; #2 \right\}} % set notation
\newcommand{\bigset}[2]{\big\{ #1 \;\big|\; #2 \big\}} % big set notation
\newcommand{\Bigset}[2]{\Big\{ #1 \;\Big|\; #2 \Big\}} % Big set notation
\newcommand{\setangle}[2]{\left\langle #1 \;\middle|\; #2 \right\rangle} % set notation
\newcommand{\ssm}{\smallsetminus} % small set minus
\newcommand{\dotprod}[2]{\left\langle \, #1 \; \middle| \; #2 \, \right\rangle} % dot product
\newcommand{\symdif}{\,\triangle\,} % symmetric difference
\newcommand{\eqdef}{\mbox{\,\raisebox{0.2ex}{\scriptsize\ensuremath{\mathrm:}}\ensuremath{=}\,}} % :=
\newcommand{\defeq}{\mbox{~\ensuremath{=}\raisebox{0.2ex}{\scriptsize\ensuremath{\mathrm:}} }} % =:
\newcommand{\simplex}{\triangle} % simplex
\renewcommand{\implies}{\Rightarrow} % imply sign
\newcommand{\transpose}[1]{{#1}^t} % transpose matrix

%%%%%%%%%%%%%%
%%  TITLE AND AUTHORS
%%%%%%%%%%%%%%
\author{Fran\c{c}ois Bergeron}

\author{Nantel Bergeron}\address[Bergeron]
{Department of Mathematics and Statistics\\ York  University\\ To\-ron\-to, Ontario M3J 1P3\\ CANADA}
\email{bergeron@mathstat.yorku.ca}
\urladdr{http://www.math.yorku.ca/bergeron}
\thanks{With partial support of Bergeron's York University Research Chair and NSERC}

\author{Cesar Caballos}
\author{Vincent Pilaud}

\title[Note and Conjectures on Multivariate Diagonal Harmonics]{Note and Conjectures on \\ Multivariate Diagonal Harmonics}

\begin{document}

\begin{abstract}
We try to summarize the advancement made in our work session in Paris, in February 2019.
\end{abstract}

\maketitle

\section{Paris, Feb 22, 2019}

\subsection{The square case~$n = m$}
We are studying the Frobenius image of the character of multivariate diagonal harmonics. Fran\c{c}ois shows that it is possible to find a formula
{\begin{equation}\label{eq:En}
{\mathcal E}_n(\Qvar;\Zvar) = \sum_{\lambda \vdash n \atop \nu \vdash \ell<{n\choose 2}} c_{\nu,\lambda} s_\nu(\Qvar) \otimes s_\lambda(\Zvar),
\end{equation}}
where the $c_{\mu,\lambda}$ are positive integers and the specialization $\Qvar \mapsto q_1 + q_2 + \dots + q_r$ gives the $r$-graded Frobenius image of the character of the multivariate diagonal harmonics with $r$ set of variables of cardinality $n$ with the diagonal action of~$S_n$.
This formula is computed explicitly by computer for~$n \in \{1,2,3,4,5,6\}$.

Guided by our heuristic that such formula such is constructed from the Tamari lattice using Hopf chain, we came to the conclusion that it is possible to write~${\mathcal E}_n(\Qvar;\Zvar)$ in different ways.
Let $\mu\subseteq\delta_n$ be a partition included in the staircase represent Dyck path in the Tamari lattice.
Possibly adding zero parts, we require that~$\ell(\mu) = n$ to encode~$n$.

\begin{conjecture}\label{conj:master}
Let $t$ be a variable, our multishuffle conjecture is
\begin{equation}\label{eq:multishuff}
\blue{
{\mathcal E}_n(\Qvar+t;\Zvar) = \sum_{\mu\subseteq\delta_n} \sigma_{\mu}(\Qvar) \otimes {\mathbb L}_{\mu}(t;\Zvar).
}
\end{equation}
Using Hopf chains, and the expression computed by Fran\c{c}ois for $n\le 6$ we arrive at the conclusion that $\sigma_{\mu}(\Qvar)$ is not Schur positive but it should be monomial positive.
In fact, for any $\mu\subseteq\delta_n$, we conjecture that it should be possible to find a subset $G_\mu$ of good chain among the Hopf chain with top element $\mu$, and a multistatistic composition ${\mathrm{mcol}}({\bf c})\models n$ for any ${\bf c}\in G_\mu$ such that
\begin{equation}\label{eq:chainstat}
\blue{
\sigma_\mu(\Qvar) = \sum_{{\bf c} \in G_\mu} M_{{\mathrm{mcol}}({\bf c})}(\Qvar),
}
\end{equation}
where $M_\alpha$ are the monomial quasisymmetric functions.
\end{conjecture}

Using some reasonable additional heuristic, we deduce the following properties of the $\sigma_\mu(\Qvar)$.
For a symmetric function $f=\sum c_\nu s_\nu$, let $\ell(f) = \max \set{\ell(\nu)}{c_\nu \ne 0}$ and $\deg(f) = \max\set{|\nu|}{c_\nu \ne 0}$.
The area of a $\mu\subseteq\delta_n$ is $\area(\mu) = {n \choose 2}-|\mu|$.

\begin{property}\label{P:len_deg}
\blue{
$\ell(\sigma_\mu) = n-|\set{k}{k \in \mu}|$} and \blue{$\deg(\sigma_\mu)\le \area(\mu)$.
}
\end{property}

Here, $|\set{k}{k\in\mu}|$ is the number of distinct parts in $\mu$ including $0$'s.
This property follows from the definition of the module of multivariate diagonal harmonics.
We also derive from the module that

\begin{property} \label{P:alt}
\blue{$\sigma_{\delta_n}=1$} and \blue{$\sigma_{0^n} = \dotprod{{\mathcal E}_n(\Qvar;\Zvar)}{s_{1^k} (\Zvar)} = {\mathcal A}_n(\Qvar)$} the $\GL$-character of the space of alternating multidiagonal harmonics.
\end{property}

We decided {\bf not to use} the property $\sigma_\mu = \sigma_{\mu'}$, since we cannot really justify it. 

\medskip
The next property is inspired from the behaviour of (Hopf) chains in the Tamari lattice. For $\mu\subseteq\delta_n$, suppose that $\mu_i=n-i$, that is the Dyck path return to the diagonal, then we can write $\mu=\alpha \odot \beta= [(n-i)^{n-i}+\beta]\cdot\alpha$, that is we decompose the dyck path of $\mu$ as the concatenation of the path of $\alpha$ followed by the path of $\beta$.

\begin{property}\label{P:multi}
If $\mu=\alpha \odot \beta$, then \blue{$\sigma_\mu=\sigma_\alpha\sigma_\beta$}.
\end{property}

We can take the scalar product of ${\mathcal E}_n(\Qvar;\Zvar)$ with different symmetric functions, this gives us identities among the $\sigma_\mu$.
It is enough to take the scalar product with Schur function (as it is a basis and we get all possible information.
For this we use the ${\mathcal E}_n(\Qvar;\Zvar)$ computed with the program $n\le 6$ in the left-hand side and use Equation~\eqref{eq:multishuff}. 

\begin{property}\label{P:coeff}
For any $\lambda \vdash n$,
\[
\blue{
\dotprod{{\mathcal E}_n(\Qvar+t;\Zvar)}{s_\lambda(\Zvar)} = \sum_{\mu\subseteq \delta_n} \dotprod{{\mathbb L}_\mu(t;\Zvar)}{s_\lambda(\Zvar)} \, \sigma_\mu(\Qvar).
}
\]
\end{property}

Notice that this gives us a property for each $\lambda\vdash n$. Our old property~{\bf 9} can be derived using~$s_{1^n}$. We have 
\begin{equation}\label{eq:alt}
\blue{
\sigma_{0^n}(\Qvar+t) =  \dotprod{{\mathcal E}_n(\Qvar+t;\Zvar)}{s_{1^n} (\Zvar)} = \sum_{\mu\subseteq \delta_n} t^{\dinv(\mu)} \sigma_\mu(\Qvar).
}
\end{equation}
This follows from the fact that $\dotprod{{\mathbb L}_\mu(t;\Zvar)}{s_\lambda(\Zvar)} = t^{\dinv(\mu)}$, since there is a unique parking function $\pi$ of shape $(\mu+1^n)/\mu$ that will give us $\comp(\pi)=1^n$ and this is the coefficient of $F_{1^n}=s_{1^n}$ which  appear only there. The $\dinv(\pi)$ of that special parking function is what we define to be $\dinv(\mu)$.

\subsection{On $\Delta$ conjecture and Tamari localization}

Now we will consider a local version of the $\Delta$-conjecture. Haglund stated that
\begin{equation}\label{eq:hag}
(\Delta'_{e_k} e_n)(q,t;\Zvar) = \sum_{\mu\subseteq\delta_n} \Big(\sum_{J\subseteq [n-1] \atop |J|=k, \, \desc(\mu)\subseteq J} q^{\sum_{i\in J} a(\mu,i)}\Big) {\mathbb L}_\mu(t;\Zvar)
\end{equation}
where $a(\mu,i)=n-i-\mu_i$ is the contribution to the area in row $i$, and~$\desc(\mu) \eqdef \set{i \in [n-1]}{\mu_i > \mu_{i+1}}$.
On the other hand, Fran\c{c}ois conjectured that~$e_{n-1-k}^{\perp_1}{\mathcal E}_n(\Qvar;\Zvar)$ should be a generalization of the $\Delta$ conjecture [\green{We have more on this later as discovered by Vincent}]. More precisely
\begin{equation}\label{eq:hag}
\Delta'_{e_k} e_n = (e_{n-1-k}^{\perp_1}{\mathcal E}_n)(q+t;\Zvar)
\end{equation}

If we expand the right-hand side using Equation~\eqref{eq:multishuff}, and use \eqref{eq:hag} for the left-hand side we get
\begin{equation}\label{eq:local}
\sum_{\mu\subseteq\delta_n} \Big(\sum_{J\subseteq [n-1] \atop |J|=k, \, \desc(\mu)\subseteq J} q^{\sum_{i\in J} a(\mu,i)}\Big) {\mathbb L}_\mu(t;\Zvar)
= \sum_{\mu\subseteq\delta_n} (e_{n-1-k}^{\perp}\sigma_{\mu})(q){\mathbb L}_{\mu}(t;\Zvar)
\end{equation}
\green{WARNING} The ${\mathbb L}_{\mu}(t;\Zvar)$ are not linearly independent but we will assume that their coefficient are equal.
Nantel justify this as the equations must be shadows of noncommutative version of LLT polynomials defined in the Catalan Hopf algebra related to Tamari lattice.
We will do the same later with non-commutative symmetric functions related to the (dual) boolean lattice.

Taking the coefficient of ${\mathbb L}_{\mu}(t;\Zvar)$ on both side of Equation~\eqref{eq:local} give what we call the Tamari localization of the $\Delta$-conjecture:

\begin{property}\label{P:localDelta}
For $\mu\subseteq\delta_n$
\[
\blue{(e_{n-1-k}^{\perp}\sigma_{\mu})(q) = \sum_{J\subseteq [n-1] \atop |J|=k, \, \desc(\mu)\subseteq J} q^{\sum_{i\in J} a(\mu,i)} 
}
\]
\end{property}

\subsection{Projection of Tamari on the (dual) boolean lattice}

For the next properties, we derive a very surprising identity from Equation~\eqref{eq:multishuff}. 
Before we need to set a few more notations.
Given $\mu \subseteq \delta_n$ we  get its descent set~$\desc(\mu) \subseteq [n-1]$.
It is well known that for a fix $A\subseteq [n-1]$, the set $\set{\mu}{\desc(\mu) = A} = [\mu_A^0, \mu_A]$ is an interval in the Tamari lattice:
\begin{itemize}
\item $\mu_A^0$ denotes the bottom of the interval, it is the bounce path defined by $A$,
\item $\mu_A$ denotes the top of the interval, it is the steep path defined by $A$.
\end{itemize}
The map $\desc$ is in fact a lattice projection from the Tamari lattice to the boolean lattice.
We sometime use composition to describe $A\subseteq[n-1]$, this has the advantage that~$n$ is understood from the composition whereas~$A$ alone does not know~$n$.
We will denote by~$c(\mu)$ the composition subset~$\desc(\mu)\subseteq [n-1]$.

Let $r_{\alpha}(\Zvar)$ be the ribbon-schur symmetric function defined by the composition $\alpha\models n$. 
For example, if $n = 6$ and $A = \{2,4,5\}$, then $\alpha = 2211$ is the composition associated to $A$.
The ribbon-Schur $r_\alpha = s_{3332/221}$ is a skew Shur function, the shape is a connected ribbon shape with $\alpha_i$ cell in row $i$.

When we let $t=0$, the polynomial LLT behave very nicely.

\begin{proposition}
\begin{equation}\label{eq:LLT0}
\blue{
{\mathbb L}_\mu(0;\Zvar) = 
\begin{cases} 0 & \text{\rm if $\mu<\mu_{\desc(\mu)}$,} \\ r_{c(\mu)}(\Zvar) & \text{\rm otherwise}.\end{cases}
}
\end{equation}
where~$\Pi_\mu$ is the set of parking function of shape~$(\mu+1^n)/u$ (standard tableau).
\end{proposition}      

\begin{proof}
Recall that the definition of LLT is
\[
{\mathbb L}_\mu(t;\Zvar) = \sum_{\pi\in \Pi_\mu} t^{\dinv(\pi)} F_{\comp(\pi)}(\Zvar)
\]
where $\dinv(\pi)$ is XXXX(hahaha), $\comp(\pi)$ is the composition associated with the descent set of the inverse of the diagonal reading the the entry of $\pi$ \red{[WARNING: do not confuse this with $c(\mu)$]} and $F$ are the Gessel fundamental quasisymmetric functions.
Now, if we compute $\dinv(\pi)$ we realize that $\dinv(\pi)>0$ always if $\mu<\mu_{\desc(\mu)}$.
That is a consequence to the fact that if we have two columns that are not consecutive than there must be at least one inversion in $\dinv(\pi)$. 

Now if we have $\mu_A$ for some $A$, then the columns of $(\mu+1^n)/u$ are consecutive. 
The parking functions of shape $(\mu+1^n)/u$ that have no dinv are exactly the filling of $(\mu+1^n)/u$ that fit the associated ribbon shape.
That shows the second part of the proposition.
\end{proof}

We can now evaluate both sides of Equation~\eqref{eq:multishuff} with $t=0$ and obtain a very surprising identity:

\begin{conjecture}\label{conj:cube}
We have an equation of ${\mathcal E}$ on the (dual) boolean lattice, namely: 
\begin{equation}\label{eq:cube}
\blue{
{\mathcal E}_n(\Qvar;\Zvar) = \sum_{A\subseteq [n-1] } \sigma_{\mu_A}(\Qvar)\otimes r_{c(\mu_A)}(\Zvar).
}
\end{equation}
\end{conjecture}

This is a totally new expression and will have many deep consequence. This should be the shadow on the (dual) boolean lattice of the expression on the Tamari lattice. 
This conjecture is a consequence of Conjecture~\ref{conj:master}. We  now put $t=1$ in Equation~\eqref{eq:multishuff}  and $\Qvar+1$ in \eqref{eq:cube}.
We have to remark that LLT symmetric function at $t=1$ is a product of elementary symmetric function, but we want to expand it as a positive sum of ribbon 
Schur function (\green{WARNING: this is not unique, we do it in a non-commutative way}). Let $B = \desc(\nu)$. We have
\[
{\mathbb L}_\nu(1;\Zvar)= e_{\alpha}=\sum_{A\subset B} r_{c(A)}(\Zvar).
\]
The second equality is a direct consequence of jeux de taquin. Let $\alpha=c(A)$, let $t=1$ in Equation~\eqref{eq:multishuff} and take the coefficient of $r_\alpha$
(\green{WARNING: the $r_\beta$ are not linearly independent but we imagine that this in written in non-commutative symmetric functions.}). We compare this to the coefficient of $r_\alpha$ in Equation~\eqref{eq:cube} with $\Qvar+1$. We get
\[
\sigma_{\mu_A}(\Qvar+1) = \sum_{\nu: A\subseteq \desc(\nu)} \sigma_\nu(\Qvar)= \sum_{\nu\le \mu_A} \sigma_\nu(\Qvar).
\]
A (dual) boolean lattice localization (coefficient of $r_{c(A)}$) at $\Qvar+1$   gives our old Property~{\bf 9$'$}:

\begin{property}\label{P:steepQ1}
For any $A\subseteq [n-1]$, let $\mu_A$ be the steep shape corresponding to $A$:
\[
\blue{\sigma_{\mu_A}(\Qvar+1) = \sum_{\nu\le \mu_A} \sigma_\nu(\Qvar)
}
\]
\end{property}

\subsection{Revisiting $\Delta$ conjecture with the (dual) boolean lattice equation}

Another use of \eqref{eq:cube} is via the following identity that Fran\c{c}ois proved:
\[
e_k^\perp \sigma_{0^n}(\Qvar) = \dotprod{{\mathcal E}_n(\Qvar;\Zvar)}{s_{(k | n-k-1)}(\Zvar)},
\]
using Frobenius notation $(k | n-k-1)$ for the hook shape partition $(k+1)1^{n-k-1}$.
We remark that
\[
\dotprod{r_{c(A)}}{s_{(k | n-k-1)}} = \delta_{|A|=k},
\]
hence taking the scalar product $\dotprod{{\mathcal E}_n(\Qvar;\Zvar)}{s_{(k | n-k-1)}(\Zvar)}$ using \eqref{eq:cube} gives

\begin{property}\label{P:DeltaAlt}
For $k\le n-1$
\[
\blue{e_k^\perp \sigma_{0^n}(\Qvar)  = \sum_{A:\  |A|=k} \sigma_{\mu_A}
}
\]
\end{property}

This is a reflection of what should be a complete general $\Delta$-conjecture.
From Fran\c{c}ois expression above and the Equation~\eqref{eq:hag}, we say that $e_{k}^{\perp_1}{\mathcal E}_n(\Qvar;\Zvar)$ should be expressible as a generalization of the $\Delta$-conjecture.
Before we do so, let us use 
now the Equation~\eqref{eq:cube} in Equation~\eqref{eq:hag}:
\begin{equation}\label{eq:deltacube}
\Delta'_{e_k} e_n = (e_{n-1-k}^{\perp_1}{\mathcal E}_n)(q,t;\Zvar)
= \sum_{A\subseteq [n-1]} ( e_{n-1-k}^\perp \sigma_{\mu_A} ) (q,t) \, r_{c(\mu_A)}
\end{equation}

\begin{property}\label{P:DeltaCube}
For $k\le n-1$
\[
\blue{
\sum_{A\subseteq [n-1]} \Big( e_{n-1-k}^\perp \sigma_{\mu_A} \Big) (q,t) \, r_{c(\mu_A)}(\Zvar)  = 
\sum_{\mu\subseteq\delta_n} \Big(\sum_{J\subseteq [n-1] \atop |J|=k, \,  \desc(\mu)\subseteq J} q^{\sum_{i\in J} a(\mu,i)}\Big) {\mathbb L}_\mu(t;\Zvar)
}
\]
\end{property}

The Property~\ref{P:localDelta} is in spirit a Tamari localization of $t=0$ in the above expression. We can also set $t=1$ and do a (dual) boolean lattice localization at $A\subseteq[n-1]$:
\begin{equation}\label{eq:cube_Delta_qt}
\blue{
\Big( e_{n-1-k}^\perp \sigma_{\mu_A} \Big) (q,1) = 
\sum_{\nu\le \mu_A} \Big(\sum_{J\subseteq [n-1] \atop |J|=k, \, \desc(\nu)\subseteq J} q^{\sum_{i\in J} a(\mu,i)}\Big)
}
\end{equation}

\subsection{Anklet and generalized $\Delta$-conjecture}

The anklet~$\ank({\bf c})$ of a Hopf chain~${\bf c} = (c_1, \dots, c_r)$ is the maximal number of paths that can be inserted between the first two paths~$c_1$ and~$c_2$ of the chain~${\bf c}$ so that it remains a Hopf chain.
Note that the anklet looks like the area of the second path~$c_2$, but it is not since the rest of the chain matters.

The anklet statistic can be decomposed according to the level of the flips.
Namely, given a Hopf chain~${\bf c} = (c_1, \dots, c_r)$, consider a Hopf chain~${\bf C}$ containing~$c$ together with $\ank({\bf c})$ additional paths in between the first two paths~$c_1$ and~$c_2$ of~${\bf c}$.
Let $\ank_p({\bf c})$ be the number of flips at the horizontal level~$p$ that were made in the part of~${\bf C}$ between the first two paths of~${\bf c}$.
This decomposes the anklet into $\ank({\bf c}) = \ank_1({\bf c}) + ... + \ank_n({\bf c})$.
(\green{WARNING: it is not clear that this decomposition is independent of the choice of the superchain~${\bf C}$.})
It yields the following generalized $\Delta$-conjecture:
\[
\blue{
(e_{n-1-k}^\perp {\mathcal E}_n)(q, t, 1^{r-2};\Zvar) = \sum_{{\bf c} \text{ good chain} \atop \text{of length } r} \sum_{J \subseteq [n-1] \atop |J| = k, \, \desc({\bf c}) \subseteq J} q^{\sum_{p \in J} \ank_p({\bf c})} \otimes {\mathbb L}_{c_r}(t;\Zvar),
}
\]
where~$\desc({\bf c}) = \set{i}{\ank_i({\bf c}) < \ank_{i-1}({\bf c})}$.
Note that this can also be computed using strict chains as follows:
\[
(e_k^\perp {\mathcal E}_n)(q, t, 1^{r-2};\Zvar) = \sum_\ell \sum_{{\bf c} \text{ strict good} \atop \text{chain of length } \ell} \sum_{J \subseteq [n-1] \atop |J| = k, \, \desc({\bf c}) \subseteq J} \!\!\!\! \bigg( \one_{k = 0} \binom{r-2}{\ell-1} + q^{\sum_{p \in J} \ank_p({\bf c})} \binom{r-2}{\ell-2} \!\! \bigg) \otimes {\mathbb L}_{c_r}(t;\Zvar).
\]
This was checked by computer for~$k < n \le 4$. For $n = 5$, we do not have the right notion of good chains.

\subsection{Some new combinatorial identities}

When we evaluate Equation~\eqref{eq:cube} at $\Qvar\mapsto q$ or $\Qvar\mapsto q+t$ we get some new combinatorial identities that one should be able to prove independently of our conjecture.
For $\Qvar\mapsto q$, the symmetric function ${\mathcal E}_n(q;\Zvar)$ is the graded action of $S_n$ on the cohomology of the Flag manifold, or the graded action of $S_n$ on the quotient of polynomials in one set of variables by the ideals of symmetric function.
This is given by the Hall-Littlewood symmetric function indexed by $1^n$.
Now evaluate Equation~\eqref{eq:cube} at $\Qvar\mapsto q$, you get
\begin{equation}\label{eq:onevar}
\blue{
\frac{h_n\big[ {\Zvar}/{(1-q)}\big]}{h_n\Big[ {1}/{(1-q)}\Big]} = \sum_{A\subseteq[n-1]} q^{\area(\mu_A)} r_{c(\mu_A)}(\Zvar).
}
\end{equation}
Is this expression known? Nantel has done this expression before (unpublished?) and think Lascoux has done it also a long time ago. 
This follow from  the shuffle theorem in $q,t$, putting $t=0$. Or equivalently, put $k=n-1$ and $t=0$ in Formula~\eqref{eq:hag}.

Now at $\Qvar\mapsto q+t$ this is {\bf really} new. Evaluating Equation~\eqref{eq:cube} at $\Qvar\mapsto q$ gives us
\begin{equation}\label{eq:onevar}
\blue{
\nabla e_n= \sum_{A\subseteq[n-1]} \sigma_{\mu_A}(q,t) r_{c(\mu_A)}.
}
\end{equation}
But even more combinatorially, if we use Equation~\eqref{eq:chainstat} evaluated at $q+t$ we get at most the two chains:
\begin{equation}\label{eq:onevar}
\sigma_{\mu_A}(q,t) = \sum_{\nu\le \mu_a} q^{\ank(\delta_n,\nu,\mu_A)} t^{\beta(\delta_n,\nu,\mu_a)}.
\end{equation}
The statistic $\ank(\delta_n,\nu,\mu_A)$ is the anklet statistic. It is almost the area of $\nu$? At least when we set $t=1$ and $k=n-1$ in Property~\ref{P:DeltaCube} we get
\[
\sigma_{\mu_A}(q,1)=\sum_{\nu\le\mu_A} q^{\area(\nu)},
\]
and this agrees with Property~\ref{P:steepQ1}. Hence we conjecture
\begin{conjecture}\label{conj:area}
Given $A\subseteq[n-1]$ and $\nu\le \mu_A$
\[
\ank(\delta_n,\nu,\mu_A)=\area(\nu).
\]
\end{conjecture}

\begin{proof}[Idea of proof]
There is a unique maximal Hopf chain $(\delta_n,c_1,\ldots,c_{\ell-1},\nu)$ such that $\ell=\area(\nu)$.
We know that chain well.
We just need to show that $(\delta_n,c_1,\ldots,c_{\ell-1},\nu,\mu_A)$ is Hopf!?
\end{proof}

Regardless of Conjecture~\ref{conj:area} being true or false, we should work hard to understand the values of $\ank$ and $\beta$ for this case and show 
\begin{equation}\label{eq:onevar}
\blue{
\nabla e_n= \sum_{A\subseteq[n-1]} \Big[\sum_{\nu\le \mu_a} q^{\ank(\delta_n,\nu,\mu_A)} t^{\beta(\delta_n,\nu,\mu_a)}\Big] r_{c(\mu_A)}.
}
\end{equation}
It would give us so much insight on the statistic we are seeking and $\nabla e_n$ is well studied. We should look at $\sigma_{\mu_A}(1,t)$...
put
\[
\sum_{A\subseteq[n-1]} \sigma_{\mu_A}(1,t) r_{c(\mu_A)}(\Zvar) = \sum_{\mu} {\mathbb L}_\mu(t;\Zvar) = (\nabla e_n)(1,t;\Zvar).
\]
The expression on the right is well known (but Nantel forgot).

\subsection{Some open questions and dreaming bigger}

\begin{enumerate}
\item Can we do localization for other lattices in between Tamari and the (dual) boolean lattice? [Nantel expec this to work if there is an associated Hopf algebra]. We can push the idea in the other direction, Is there an analogue of Formula~\eqref{eq:multishuff} over permutations? We can then localize over the Bruhat order? That would give us even more refined properties
\item Our exploration suggests that we should be able to construct the $\sigma_\mu$ via operators applied to $\bf 1$. These operators could be ``dual" to the Carlson-Mellit operator?
\item We feel there should be a collar version of dinv
\item We have to understand better what happens when the variable move from one side of the tensor to the other side in the expression ${\mathcal E}_n(\Qvar;\Zvar)$. Fran\c{c}ois' experiments suggest (explicit expressions for $n\leq 4$, and partial for $n=5$) that there is in fact a nice expansion of the form
\begin{equation}
    \mathcal{E}_n(\bm{q}+\bm{t};\bm{z}) = \sum_{\mu\subseteq \delta_n} \sigma_\mu(\bm{q}) \mathbb{L}_\mu(\bm{t};\bm{z}),
  \end{equation}
  with $\bm{t}$ a general set of variables. To get these it is interesting to exploit the specialization that corresponds to  Conjecture~\ref{conj:master} (all left terms of length $1$ in $\mathbb{L}_\mu(\bm{t};\bm{z})$ correspond to classical LLT: $\mathbb{L}_\mu(t;\bm{z})$), as well as the fact that 
  the formula has to be $(\bm{q},\bm{t})$-symmetric, so that one has 
     $$\sum_{\mu\subseteq \delta_n} \sigma_\mu(\bm{q}) \mathbb{L}_\mu(\bm{t};\bm{z})=\sum_{\mu\subseteq \delta_n} \sigma_\mu(\bm{t}) \mathbb{L}_\mu(\bm{q};\bm{z}).$$
     Calculation suggest that there are nice stabilities (to be described later) in the ``coefficients''.
\end{enumerate}

%-------------------------------------------------------------------
% Begin BIBLIOGRAPHY
%-------------------------------------------------------------------

%\small
%\bibliographystyle{abbrv}  
%%\bibliographystyle{amsalpha}  
%%\bibliographystyle{plain}  
%%\bibliographystyle{amsplain}  
%\bibliography{}

\end{document}
